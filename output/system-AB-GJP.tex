\documentclass[a4paper]{article}
\usepackage[margin=1in]{geometry}
\usepackage[T1]{fontenc}
\usepackage[utf8]{inputenc}
\usepackage{newcent}
\usepackage{helvet}
\usepackage{graphicx}
\usepackage{dirtree}
\usepackage[pdftex, pdfborder={0 0 0}]{hyperref}
\frenchspacing

\usepackage{txfonts} % For \varheartsuit and \vardiamondsuit
\usepackage[usenames,dvipsnames]{color} % dvipsnames necessary to made PDFLaTeX work.
\usepackage{enumerate}
\usepackage{listliketab}
\usepackage{latexsym} % \Box
\usepackage{pbox} % \Box
\usepackage{parskip} % line between paragraphs

% suits

%%% Colors

\newcommand{\BC}{\textcolor{OliveGreen}{$\clubsuit$}}
\newcommand{\BD}{\textcolor{RedOrange}{$\vardiamondsuit$}}
\newcommand{\BH}{\textcolor{Red}{$\varheartsuit${}}}
\newcommand{\BS}{\textcolor{Blue}{$\spadesuit${}}}

%suits for pdf-friendly titles
\newcommand{\pdfc}{\texorpdfstring{\BC{}}{C}}
\newcommand{\pdfd}{\texorpdfstring{\BD{}}{D}}
\newcommand{\pdfh}{\texorpdfstring{\BH{}}{H}}
\newcommand{\pdfs}{\texorpdfstring{\BS{}}{S}}

\newenvironment{bidtable}
{\begin{tabbing}

xxxxxx\=xxxxxx\=xxxxxx\=xxxxxx\=xxxxxx\=xxxxxx\=xxxxxx\=xxxxxx\=xxxxxx\=xxxxxx\=\kill}
{\end{tabbing} }%

\newenvironment{bidding}%
{\begin{tabbing}
xxxxxx\=xxxxxx\=xxxxxx\=xxxxxx \kill
}{\end{tabbing} }%end bidding


% writing hands
\newcommand{\cards}[1]{\textsf{#1}}
\newcommand{\spades}[1]{\BS\cards{#1}}
\newcommand{\hearts}[1]{\BH\cards{#1}}
\newcommand{\diamonds}[1]{\BD\cards{#1}}
\newcommand{\clubs}[1]{\BC\cards{#1}}
\newcommand{\void}{--}
\newcommand{\hand}[4]{\spades{#1}\ \hearts{#2}\ \diamonds{#3} \clubs{#4}}
\newcommand{\vhand}[4]{\spades{#1}\\\hearts{#2}\\\diamonds{#3}\\\clubs{#4}}

% The \Box should always appear the same distance from the left margin
\newcommand\onesuit[4]%
{%
  \begin{center}%
    \begin{tabular}{>{\hfill}p{3cm}cp{3cm}}
                & \cards{#2} \\%
      \cards{#1}& $\Box$    & \cards{#3} \\%
                & \cards{#4} %
    \end{tabular}
  \end{center}%
}

% A special command if the south hand is not shown to avoid whitespace
\newcommand\onesuitenw[3]%
{%
  \begin{center}%
    \begin{tabular}{>{\hfill}p{3cm}cp{3cm}}%
                & \cards{#2} \\%
      \cards{#1}& $\Box$    & \cards{#3}%
    \end{tabular}%
  \end{center}%
}

\newcommand\dealdiagram[5]%
{%
  \begin{center}%
    \begin{tabular}{>{\hfill}p{3cm}cp{3cm}}
      \pbox{20cm}{\small #5}& \pbox{20cm}{#2} \\%
      \pbox{20cm}{#1}& $\Box$    & \pbox{20cm}{#3} \\%
                & \pbox{20cm}{#4} %
    \end{tabular}
  \end{center}%
}

\newcommand\dealdiagramenw[4]%
{%
  \begin{center}%
    \begin{tabular}{>{\hfill}p{3cm}cp{3cm}}
      \pbox{20cm}{\small #4}& \pbox{20cm}{#2} \\%
      \pbox{20cm}{#1}& $\Box$    & \pbox{20cm}{#3} \\%
    \end{tabular}
  \end{center}%
}

\newcommand\dealdiagramew[2]%
{%
  \begin{center}%
    \begin{tabular}{>{\hfill}p{3cm}cp{3cm}}
      \pbox{20cm}{#1}& $\Box$    & \pbox{20cm}{#2} \\%
    \end{tabular}
  \end{center}%
}

\title{AB-GJP 1.1}
\author{Albert Bitran / Gert-Jan Paulissen}
\begin{document}
\maketitle
\tableofcontents

\section{Introduction}

\subsection{BML - Bridge Bidding Markup Language}

The notation in this document is in BML format, see
github.com/gpaulissen/bml/blob/master/README.org.
\bigbreak
This notation allows for text documents with simple bidding tables.
\bigbreak
\subsubsection{Output}

Using a simple BML document one can generate:
\bigbreak
\begin{itemize}
\item a PDF file

\item an HTML file

\item a BSS file, a BBO system file, which can be uploaded using the BBO desktop Windows client.

\end{itemize}
\bigbreak
\subsection{Abbreviations}

The abbreviations used are mostly from the WBF with some changes, see also the
Abbreviations chapter at the end.
\bigbreak
\section{Opening bids}

Let's start with the basic opening structure of the system (:
\bigbreak
\
\dirtree{%%
 .1 \O. 
 .2 1\BC......\begin{minipage}[t]{0.8\textwidth}
2+\BC , NAT or BAL, 11+ HCP (12+ if BAL)
\end{minipage}. 
 .2 1\BD......\begin{minipage}[t]{0.8\textwidth}
4+\BD , 11+ HCP (12+ if BAL)
\end{minipage}. 
 .2 1\BH\BS.....\begin{minipage}[t]{0.8\textwidth}
5+\BH /\BS , 11+ HCP (12+ if BAL)
\end{minipage}. 
 .2 1NT.....\begin{minipage}[t]{0.8\textwidth}
15--17 BAL
\end{minipage}. 
 .2 2M......\begin{minipage}[t]{0.8\textwidth}
5+M, weak-two, vulnerable 6 cards and quite a good hand
\end{minipage}. 
 .2 2NT.....\begin{minipage}[t]{0.8\textwidth}
20-22 HCP, BAL (5M332 or 5m422 possible) and maybe even 4441, 5431 or 6322 (no 5+M)
\end{minipage}. 
 .2 3X......\begin{minipage}[t]{0.8\textwidth}
PRE
\end{minipage}. 
 .2 3NT.....\begin{minipage}[t]{0.8\textwidth}
Gambling
\end{minipage}. 
}
\bigbreak
\section{The 1 major opening}

One of a major promises 5 cards and at least 11 HCP. It is limited since with
game forcing hands you open at the two level.
\bigbreak
\subsection{Answers}

\
\dirtree{%%
 .1 1\BH. 
 .2 1\BS......\begin{minipage}[t]{0.8\textwidth}
4\BS\ and normally 6+ HCP
\end{minipage}. 
 .2 1NT.....\begin{minipage}[t]{0.8\textwidth}
no 4\BS , 5-12 HCP, with fit you have a minimum
\end{minipage}. 
 .2 2\BC......\begin{minipage}[t]{0.8\textwidth}
FG
\end{minipage}. 
 .2 2\BD......\begin{minipage}[t]{0.8\textwidth}
5+\BD , FG
\end{minipage}. 
 .2 2\BH......\begin{minipage}[t]{0.8\textwidth}
3\BH , 7-9 HCP
\end{minipage}. 
 .2 2\BS......\begin{minipage}[t]{0.8\textwidth}
6\BS , 8-11 HCP
\end{minipage}. 
 .2 2NT.....\begin{minipage}[t]{0.8\textwidth}
fit, 10-14 HCP
\end{minipage}. 
 .2 3m......\begin{minipage}[t]{0.8\textwidth}
6m, INV
\end{minipage}. 
}
\bigbreak
\
\dirtree{%%
 .1 1\BS. 
 .2 1NT.....\begin{minipage}[t]{0.8\textwidth}
5-12 HCP, with fit you have a minimum
\end{minipage}. 
 .2 2\BC......\begin{minipage}[t]{0.8\textwidth}
FG
\end{minipage}. 
 .2 2\BD......\begin{minipage}[t]{0.8\textwidth}
5+\BD , FG
\end{minipage}. 
 .2 2\BH......\begin{minipage}[t]{0.8\textwidth}
5+\BH , FG
\end{minipage}. 
 .2 2\BS......\begin{minipage}[t]{0.8\textwidth}
3\BS , 7-9 HCP
\end{minipage}. 
 .2 2NT.....\begin{minipage}[t]{0.8\textwidth}
fit, 10-14 HCP
\end{minipage}. 
 .2 3m......\begin{minipage}[t]{0.8\textwidth}
6m, INV
\end{minipage}. 
 .2 3\BH......\begin{minipage}[t]{0.8\textwidth}
6\BH , INV
\end{minipage}. 
}
\bigbreak
\section{Fit after 1M}

\subsection{Answers}

\
\dirtree{%%
 .1 1M. 
 .2 2M......\begin{minipage}[t]{0.8\textwidth}
fit, 7-9 (10) HCP
\end{minipage}. 
 .3 3M......\begin{minipage}[t]{0.8\textwidth}
(mild) INV, normally 6M
\end{minipage}. 
 .3 2\BS/3\BH...\begin{minipage}[t]{0.8\textwidth}
NAT, F
\end{minipage}. 
 .3 2NT.....\begin{minipage}[t]{0.8\textwidth}
limit
\end{minipage}. 
 .3 3m......\begin{minipage}[t]{0.8\textwidth}
NAT, F
\end{minipage}. 
 .3 4M......\begin{minipage}[t]{0.8\textwidth}
S/O
\end{minipage}. 
 .3 4m......\begin{minipage}[t]{0.8\textwidth}
SPL, S/T
\end{minipage}. 
 .2 2NT.....\begin{minipage}[t]{0.8\textwidth}
INV, at least 3 cards fit
\end{minipage}. 
 .3 3\BC\BD.....\begin{minipage}[t]{0.8\textwidth}
?
\end{minipage}. 
 .3 3oM.....\begin{minipage}[t]{0.8\textwidth}
?
\end{minipage}. 
 .2 3M......\begin{minipage}[t]{0.8\textwidth}
at least 4 cards fit and a SPL, 8-9 HCP
\end{minipage}. 
 .2 3NT.....\begin{minipage}[t]{0.8\textwidth}
fit but very weak (?)
\end{minipage}. 
 .2 3\BC......\begin{minipage}[t]{0.8\textwidth}
at least 4 cards fit, 12-15 HCP, no SPL
\end{minipage}. 
 .2 3\BD......\begin{minipage}[t]{0.8\textwidth}
3 cards fit, 12-15 HCP, no SPL
\end{minipage}. 
 .2 4m......\begin{minipage}[t]{0.8\textwidth}
SPL, 4 cards fit, 11-12 HCP
\end{minipage}. 
 .2 4M......\begin{minipage}[t]{0.8\textwidth}
T/P, expect to win
\end{minipage}. 
}
\bigbreak
\
\dirtree{%%
 .1 1\BH. 
 .2 3\BS......\begin{minipage}[t]{0.8\textwidth}
SPL, 4 cards fit, 11-12 HCP
\end{minipage}. 
}
\bigbreak
\
\dirtree{%%
 .1 1\BS. 
 .2 4\BH......\begin{minipage}[t]{0.8\textwidth}
SPL, 4 cards fit, 11-12 HCP (NON!!!????)
\end{minipage}. 
}
\bigbreak
\subsection{Remarks by GJP}

Too much space is used by all fit bids. Two over one and invitational hands
like Ax, xxx, xx, KQJ10xx will be too difficult to bid.
\bigbreak
\subsection{Passed hand bidding}

We play 2\BC\ Drury after an opening in third or fourt hand. Opener returns to
his major when weak (or bids 2\BH\ after 1\BS ). 2\BD\ is waiting and positive. Bids
above 2M by opener are forcing game.
\bigbreak
A single jump in a new suit after 1M is not a fit bid (so no ``annonce de rencontre''). We
never play single jumps as fit bids.
\bigbreak
\subsection{2NT after intervention}

Always natural, not fit like without intervention.
\bigbreak
\section{The 1NT Opening}

Shows 15-17 HCP (semi)Balanced, 5m422 or 6m are possible and rarely a singleton
K or Q. 14 or 17 with a 5crd suit are upgraded regularly.
\bigbreak
\subsection{Answers}

\
\dirtree{%%
 .1 1NT. 
 .2 2\BC......\begin{minipage}[t]{0.8\textwidth}
Stayman, four responses
\end{minipage}. 
 .2 2\BD......\begin{minipage}[t]{0.8\textwidth}
5+\BH\ TRF
\end{minipage}. 
 .2 2\BH......\begin{minipage}[t]{0.8\textwidth}
5+\BS\ TRF
\end{minipage}. 
 .2 2\BS......\begin{minipage}[t]{0.8\textwidth}
5+\BC\ TRF
\end{minipage}. 
 .2 2NT.....\begin{minipage}[t]{0.8\textwidth}
8-9 HCP, BAL
\end{minipage}. 
 .2 3\BC......\begin{minipage}[t]{0.8\textwidth}
6+\BD\ TRF
\end{minipage}. 
 .2 3\BD......\begin{minipage}[t]{0.8\textwidth}
55 MM, INV
\end{minipage}. 
 .2 3\BH......\begin{minipage}[t]{0.8\textwidth}
31(54) FG
\end{minipage}. 
 .2 3\BS......\begin{minipage}[t]{0.8\textwidth}
13(54) FG
\end{minipage}. 
 .2 3NT.....\begin{minipage}[t]{0.8\textwidth}
T/P
\end{minipage}. 
 .2 4\BC......\begin{minipage}[t]{0.8\textwidth}
6\BH 5\BS 
\end{minipage}. 
 .2 4\BD......\begin{minipage}[t]{0.8\textwidth}
55 MM
\end{minipage}. 
 .2 4\BC......\begin{minipage}[t]{0.8\textwidth}
5\BH 6\BS 
\end{minipage}. 
 .2 4M......\begin{minipage}[t]{0.8\textwidth}
T/P
\end{minipage}. 
 .2 4NT.....\begin{minipage}[t]{0.8\textwidth}
QUANT
\end{minipage}. 
 .2 5m......\begin{minipage}[t]{0.8\textwidth}
T/P
\end{minipage}. 
}
\bigbreak
\subsubsection{Continuation after Stayman}

\
\dirtree{%%
 .1 1NT-2\BC. 
 .2 2\BD......\begin{minipage}[t]{0.8\textwidth}
no 4M
\end{minipage}. 
 .2 2\BH......\begin{minipage}[t]{0.8\textwidth}
4+\BH , no 4\BS 
\end{minipage}. 
 .2 2\BS......\begin{minipage}[t]{0.8\textwidth}
4+\BS , no 4\BH 
\end{minipage}. 
 .2 2NT.....\begin{minipage}[t]{0.8\textwidth}
44MM
\end{minipage}. 
 .3 3\BC......\begin{minipage}[t]{0.8\textwidth}
4+\BH , INV or S/T
\end{minipage}. 
 .3 3\BD......\begin{minipage}[t]{0.8\textwidth}
4+\BS , INV or S/T
\end{minipage}. 
 .3 4\BC......\begin{minipage}[t]{0.8\textwidth}
4+\BH , GF
\end{minipage}. 
 .3 4\BD......\begin{minipage}[t]{0.8\textwidth}
4+\BS , GF
\end{minipage}. 
}
\bigbreak
\
\dirtree{%%
 .1 1NT-2\BC-2\BD. 
 .2 2\BH......\begin{minipage}[t]{0.8\textwidth}
5+\BH , INV (misère dorée)
\end{minipage}. 
 .2 2\BS......\begin{minipage}[t]{0.8\textwidth}
5+\BS , INV (misère dorée)
\end{minipage}. 
 .2 2NT.....\begin{minipage}[t]{0.8\textwidth}
INV
\end{minipage}. 
 .2 3m......\begin{minipage}[t]{0.8\textwidth}
5+m 4M FG
\end{minipage}. 
 .2 3\BH......\begin{minipage}[t]{0.8\textwidth}
Smolen (chassé croisé) with 5\BS -4\BH , FG
\end{minipage}. 
 .2 3\BS......\begin{minipage}[t]{0.8\textwidth}
Smolen (chassé croisé) with 5\BH -4\BS , FG
\end{minipage}. 
 .2 3NT.....\begin{minipage}[t]{0.8\textwidth}
T/P
\end{minipage}. 
 .2 4\BC......\begin{minipage}[t]{0.8\textwidth}
Smolen (chassé croisé) with 6\BH -4\BS , FG
\end{minipage}. 
 .2 4\BD......\begin{minipage}[t]{0.8\textwidth}
Smolen (chassé croisé) with 6\BS -4\BH , FG
\end{minipage}. 
 .2 4NT.....\begin{minipage}[t]{0.8\textwidth}
QUANT
\end{minipage}. 
}
\bigbreak
\
\dirtree{%%
 .1 1NT-2\BC-2\BH. 
 .2 2\BS......\begin{minipage}[t]{0.8\textwidth}
5+\BS , INV (misère dorée)
\end{minipage}. 
 .2 2NT.....\begin{minipage}[t]{0.8\textwidth}
INV
\end{minipage}. 
 .2 3m......\begin{minipage}[t]{0.8\textwidth}
5+m-4\BS\ FG
\end{minipage}. 
 .2 3\BH......\begin{minipage}[t]{0.8\textwidth}
INV
\end{minipage}. 
 .2 3\BS......\begin{minipage}[t]{0.8\textwidth}
4+\BH , S/T
\end{minipage}. 
 .3 3NT.....\begin{minipage}[t]{0.8\textwidth}
CTRL \BS 
\end{minipage}. 
 .2 4m......\begin{minipage}[t]{0.8\textwidth}
4+\BH , S/T, SPL m
\end{minipage}. 
 .2 4NT.....\begin{minipage}[t]{0.8\textwidth}
QUANT
\end{minipage}. 
 .2 4\BS......\begin{minipage}[t]{0.8\textwidth}
4+\BH , BW exclusion
\end{minipage}. 
 .2 5m......\begin{minipage}[t]{0.8\textwidth}
4+\BH , BW exclusion
\end{minipage}. 
}
\bigbreak
\
\dirtree{%%
 .1 1NT-2\BC-2\BS. 
 .2 2NT.....\begin{minipage}[t]{0.8\textwidth}
INV
\end{minipage}. 
 .3 3\BH......\begin{minipage}[t]{0.8\textwidth}
3\BH , GF, to play game opposite 5\BH\ (misère dorée)
\end{minipage}. 
 .2 3m......\begin{minipage}[t]{0.8\textwidth}
5+m-4\BH\ FG
\end{minipage}. 
 .2 3\BH......\begin{minipage}[t]{0.8\textwidth}
4+\BS , S/T
\end{minipage}. 
 .2 3\BS......\begin{minipage}[t]{0.8\textwidth}
INV
\end{minipage}. 
 .2 4\BC\BD\BH....\begin{minipage}[t]{0.8\textwidth}
SPL, 4+\BS , S/T
\end{minipage}. 
 .2 5\BC\BD\BH....\begin{minipage}[t]{0.8\textwidth}
4+\BS , BW exclusion
\end{minipage}. 
}
\bigbreak
\subsubsection{Continuation after Transfers}

\
\dirtree{%%
 .1 1NT-2\BD. 
 .2 2\BH......\begin{minipage}[t]{0.8\textwidth}
NF
\end{minipage}. 
 .3 2\BS......\begin{minipage}[t]{0.8\textwidth}
55 MM, INV (et 1N-3D ?)
\end{minipage}. 
 .3 2NT.....\begin{minipage}[t]{0.8\textwidth}
FG, 5M4m(31), no S/T
\end{minipage}. 
 .4 3\BC......\begin{minipage}[t]{0.8\textwidth}
(R)
\end{minipage}. 
 .5 3\BD......\begin{minipage}[t]{0.8\textwidth}
4\BC , 3\BD 
\end{minipage}. 
 .5 3\BH......\begin{minipage}[t]{0.8\textwidth}
4\BC , 1\BD 
\end{minipage}. 
 .5 3\BS......\begin{minipage}[t]{0.8\textwidth}
4\BD , 3\BC 
\end{minipage}. 
 .5 3NT.....\begin{minipage}[t]{0.8\textwidth}
4\BD , 1\BC 
\end{minipage}. 
 .3 3m......\begin{minipage}[t]{0.8\textwidth}
5+m, FG or 4m, S/T
\end{minipage}. 
 .4 3M......\begin{minipage}[t]{0.8\textwidth}
might be ``I am stuck''
\end{minipage}. 
 .3 3\BH......\begin{minipage}[t]{0.8\textwidth}
6+\BH\ INV
\end{minipage}. 
 .3 3\BS......\begin{minipage}[t]{0.8\textwidth}
6+\BH\ SPL \BS\ or no SPL
\end{minipage}. 
 .3 3NT.....\begin{minipage}[t]{0.8\textwidth}
T/P
\end{minipage}. 
 .3 4m......\begin{minipage}[t]{0.8\textwidth}
6+\BH\ SPL
\end{minipage}. 
 .3 4\BH......\begin{minipage}[t]{0.8\textwidth}
T/P
\end{minipage}. 
 .2 2NT.....\begin{minipage}[t]{0.8\textwidth}
4+\BH\ MAX
\end{minipage}. 
 .3 3\BC......\begin{minipage}[t]{0.8\textwidth}
INV
\end{minipage}. 
 .3 3\BD......\begin{minipage}[t]{0.8\textwidth}
TRF
\end{minipage}. 
 .3 3\BS......\begin{minipage}[t]{0.8\textwidth}
SPL
\end{minipage}. 
 .3 4m......\begin{minipage}[t]{0.8\textwidth}
SPL
\end{minipage}. 
 .2 3\BH......\begin{minipage}[t]{0.8\textwidth}
4+\BH\ MIN
\end{minipage}. 
}
\bigbreak
After 1NT-2\BH\ we use the same structure as over 1NT-2\BD , so very natural. Some exceptions:
\bigbreak
\
\dirtree{%%
 .1 1NT-2\BH. 
 .2 2\BS......\begin{minipage}[t]{0.8\textwidth}
NF
\end{minipage}. 
 .3 3\BH......\begin{minipage}[t]{0.8\textwidth}
55 MM, S/T
\end{minipage}. 
 .3 4\BC......\begin{minipage}[t]{0.8\textwidth}
6\BS , S/T, SPL \BC\ or no SPL
\end{minipage}. 
}
\bigbreak
\
\dirtree{%%
 .1 1NT-2\BS..\begin{minipage}[t]{0.8\textwidth}
5+\BC 
\end{minipage}. 
 .2 2NT.....\begin{minipage}[t]{0.8\textwidth}
fit \BC , INV
\end{minipage}. 
 .3 3\BC......\begin{minipage}[t]{0.8\textwidth}
T/P
\end{minipage}. 
 .3 3\BD......\begin{minipage}[t]{0.8\textwidth}
55 mm, FG
\end{minipage}. 
 .3 3M......\begin{minipage}[t]{0.8\textwidth}
6\BC , SPL
\end{minipage}. 
 .3 3NT.....\begin{minipage}[t]{0.8\textwidth}
accept INV
\end{minipage}. 
 .3 4\BC......\begin{minipage}[t]{0.8\textwidth}
S/T, no SPL M
\end{minipage}. 
 .3 4\BD\BH\BS....\begin{minipage}[t]{0.8\textwidth}
BW exclusion
\end{minipage}. 
 .2 3\BC......\begin{minipage}[t]{0.8\textwidth}
else
\end{minipage}. 
 .3 3NT.....\begin{minipage}[t]{0.8\textwidth}
suggests SPL \BD 
\end{minipage}. 
}
\bigbreak
\
\dirtree{%%
 .1 1NT-3\BC. 
 .2 3\BD......\begin{minipage}[t]{0.8\textwidth}
Forced
\end{minipage}. 
 .3 3M......\begin{minipage}[t]{0.8\textwidth}
SPL
\end{minipage}. 
 .3 3NT.....\begin{minipage}[t]{0.8\textwidth}
suggest SPL \BC 
\end{minipage}. 
 .3 4\BC......\begin{minipage}[t]{0.8\textwidth}
SPL \BC , S/T
\end{minipage}. 
 .3 4\BD......\begin{minipage}[t]{0.8\textwidth}
S/T, no SPL
\end{minipage}. 
 .3 4\BH\BS.....\begin{minipage}[t]{0.8\textwidth}
BW exclusion
\end{minipage}. 
 .3 4\BC......\begin{minipage}[t]{0.8\textwidth}
BW exclusion
\end{minipage}. 
}
\bigbreak
\subsection{They bid over our 1NT opening}

After a double we play system on. The only new bid is redouble which depends
on the meaning of double. If double is for penalties (or just showing values),
the redouble is a Puppet to 2\BC . In any other case the redouble shows values
as well and is forcing till 2\BS .
\bigbreak
\
\dirtree{%%
 .1 1NT-(2\BC). 
 .2 Pass....\begin{minipage}[t]{0.8\textwidth}
weak or a trap pass (for penalties)
\end{minipage}. 
 .2 Dbl.....\begin{minipage}[t]{0.8\textwidth}
values and a double later on is for take out
\end{minipage}. 
 .2 2\BD......\begin{minipage}[t]{0.8\textwidth}
NF
\end{minipage}. 
 .2 2\BH......\begin{minipage}[t]{0.8\textwidth}
NF (?)
\end{minipage}. 
 .2 2\BS......\begin{minipage}[t]{0.8\textwidth}
minors (?)
\end{minipage}. 
 .2 2NT.....\begin{minipage}[t]{0.8\textwidth}
transfer lebensohl
\end{minipage}. 
 .2 3X......\begin{minipage}[t]{0.8\textwidth}
transfer lebensohl
\end{minipage}. 
}
\bigbreak
\subsubsection{They bid over stayman}

\
\dirtree{%%
 .1 1NT-(P)-2\BC-(D). 
 .2 Pass....\begin{minipage}[t]{0.8\textwidth}
no \BC\ stopper
\end{minipage}. 
 .3 Rdbl....\begin{minipage}[t]{0.8\textwidth}
Stayman again, INV+
\end{minipage}. 
 .4 2\BH......\begin{minipage}[t]{0.8\textwidth}
4+\BS 
\end{minipage}. 
 .4 2\BS......\begin{minipage}[t]{0.8\textwidth}
4+\BH 
\end{minipage}. 
 .4 2\BD......\begin{minipage}[t]{0.8\textwidth}
no 4+M
\end{minipage}. 
 .4 2NT.....\begin{minipage}[t]{0.8\textwidth}
MM, MIN
\end{minipage}. 
 .4 3\BC......\begin{minipage}[t]{0.8\textwidth}
MM, MAX
\end{minipage}. 
 .2 Rdbl....\begin{minipage}[t]{0.8\textwidth}
Proposal to play (4)5+\BC 
\end{minipage}. 
 .2 2\BD\BH\BS....\begin{minipage}[t]{0.8\textwidth}
\BC\ stopper, system on
\end{minipage}. 
 .2 2NT.....\begin{minipage}[t]{0.8\textwidth}
\BC\ stopper, system on
\end{minipage}. 
 .2 3\BC......\begin{minipage}[t]{0.8\textwidth}
\BC\ stopper, system on
\end{minipage}. 
}
\bigbreak
\
\dirtree{%%
 .1 1NT-(P)-2\BC-(2\BD). 
 .2 Pass....\begin{minipage}[t]{0.8\textwidth}
no M
\end{minipage}. 
 .2 Dbl.....\begin{minipage}[t]{0.8\textwidth}
for penalties
\end{minipage}. 
 .2 2M......\begin{minipage}[t]{0.8\textwidth}
4+M
\end{minipage}. 
 .2 2NT.....\begin{minipage}[t]{0.8\textwidth}
MM, MIN
\end{minipage}. 
 .2 3\BC......\begin{minipage}[t]{0.8\textwidth}
MM, MAX
\end{minipage}. 
}
\bigbreak
\
\dirtree{%%
 .1 1NT-(P)-2\BC-(2M). 
 .2 Dbl.....\begin{minipage}[t]{0.8\textwidth}
take out (4+oM)
\end{minipage}. 
 .2 2\BS......\begin{minipage}[t]{0.8\textwidth}
5+\BS 
\end{minipage}. 
}
\bigbreak
\
\dirtree{%%
 .1 1NT-(P)-2\BC-(3m). 
 .2 Dbl.....\begin{minipage}[t]{0.8\textwidth}
at least one major
\end{minipage}. 
 .2 3M......\begin{minipage}[t]{0.8\textwidth}
5+M
\end{minipage}. 
}
\bigbreak
\subsubsection{They bid over our transfer}

\
\dirtree{%%
 .1 1NT-(P)-2red-(D). 
 .2 Pass....\begin{minipage}[t]{0.8\textwidth}
no 3 cards fit
\end{minipage}. 
 .3 Rdbl....\begin{minipage}[t]{0.8\textwidth}
retransfer
\end{minipage}. 
 .3 1step...\begin{minipage}[t]{0.8\textwidth}
to play
\end{minipage}. 
 .3 3m......\begin{minipage}[t]{0.8\textwidth}
5-5, NF
\end{minipage}. 
 .2 Rdbl....\begin{minipage}[t]{0.8\textwidth}
3+ cards fit, wants partner to play (usually no stopper or something like Ax(x))
\end{minipage}. 
 .2 2M......\begin{minipage}[t]{0.8\textwidth}
3+ cards fit, wants to play (usually a stopper)
\end{minipage}. 
}
\bigbreak
When they bid over our transfer a double is just for penalties.  Support shows
a good hand with fit, after 2D at least 4 cards (partner may have 4\BH 5\BS\ in
the majors). If they bid our cuebid dbl shows fit and willingness to compete.
\bigbreak
\section{Weak-two opening bids}

Both 2\BH\ and 2\BS\ show a weak two bid and less than a one level opening
bid. Non-vulnerable possibly 5M (especially by Albert) and usually 5-10
HCP. Vulnerable the opening promises a good 6 card suit and 9-11 HCP.
\bigbreak
\subsection{Answers}

A change of suits is non forcing after a non-vulnerable opening but forcing otherwise.
\bigbreak
\subsubsection{2M-2NT}

Classic French, so a bid shows a top honor and a repetition is the weakest bid.
\bigbreak
\section{The 2NT opening}

\subsection{Answers}

\
\dirtree{%%
 .1 2NT. 
 .2 3\BC......\begin{minipage}[t]{0.8\textwidth}
Stayman, four responses
\end{minipage}. 
 .3 3\BD......\begin{minipage}[t]{0.8\textwidth}
no 4M
\end{minipage}. 
 .4 3\BH......\begin{minipage}[t]{0.8\textwidth}
Smolen (chassé croisé) with 5\BS -4\BH , FG
\end{minipage}. 
 .4 3\BS......\begin{minipage}[t]{0.8\textwidth}
Smolen (chassé croisé) with 5\BH -4\BS , FG
\end{minipage}. 
 .3 3\BH......\begin{minipage}[t]{0.8\textwidth}
4+\BH , no 4\BS 
\end{minipage}. 
 .4 3\BS......\begin{minipage}[t]{0.8\textwidth}
\BH\ fit, S/T
\end{minipage}. 
 .3 3\BS......\begin{minipage}[t]{0.8\textwidth}
4+\BS , no 4\BH 
\end{minipage}. 
 .4 4\BH......\begin{minipage}[t]{0.8\textwidth}
\BS\ fit, S/T
\end{minipage}. 
 .3 3NT.....\begin{minipage}[t]{0.8\textwidth}
44 MM
\end{minipage}. 
 .4 4\BC......\begin{minipage}[t]{0.8\textwidth}
TRF
\end{minipage}. 
 .4 4\BD......\begin{minipage}[t]{0.8\textwidth}
TRF
\end{minipage}. 
 .4 4M......\begin{minipage}[t]{0.8\textwidth}
4M, S/T, NF
\end{minipage}. 
 .2 3\BD......\begin{minipage}[t]{0.8\textwidth}
5+\BH , TRF
\end{minipage}. 
 .3 3\BH......\begin{minipage}[t]{0.8\textwidth}
NF (rectification non fittée)
\end{minipage}. 
 .4 3\BS......\begin{minipage}[t]{0.8\textwidth}
55 MM, S/T
\end{minipage}. 
 .5 3NT.....\begin{minipage}[t]{0.8\textwidth}
T/P
\end{minipage}. 
 .5 4\BC......\begin{minipage}[t]{0.8\textwidth}
\BH\ fit
\end{minipage}. 
 .5 4\BD......\begin{minipage}[t]{0.8\textwidth}
\BS\ fit
\end{minipage}. 
 .4 4\BC......\begin{minipage}[t]{0.8\textwidth}
5\BH -4\BC 
\end{minipage}. 
 .5 4\BD......\begin{minipage}[t]{0.8\textwidth}
CTRL for \BC\ (only great fit for \BC\ possible)
\end{minipage}. 
 .5 4\BH......\begin{minipage}[t]{0.8\textwidth}
NAT
\end{minipage}. 
 .4 4\BD......\begin{minipage}[t]{0.8\textwidth}
5\BH -4\BD 
\end{minipage}. 
 .5 4\BH......\begin{minipage}[t]{0.8\textwidth}
NAT
\end{minipage}. 
 .5 4\BS......\begin{minipage}[t]{0.8\textwidth}
CTRL for \BD\ (only great fit for \BD\ possible)
\end{minipage}. 
 .3 3\BS......\begin{minipage}[t]{0.8\textwidth}
good fit, second suit
\end{minipage}. 
 .3 3NT.....\begin{minipage}[t]{0.8\textwidth}
3crd fit
\end{minipage}. 
 .3 4m......\begin{minipage}[t]{0.8\textwidth}
good fit, second suit
\end{minipage}. 
 .2 3\BH......\begin{minipage}[t]{0.8\textwidth}
5+\BS , TRF
\end{minipage}. 
 .3 3\BS......\begin{minipage}[t]{0.8\textwidth}
NF (rectification non fittée)
\end{minipage}. 
 .4 4\BC......\begin{minipage}[t]{0.8\textwidth}
5\BS -4\BC 
\end{minipage}. 
 .4 4\BD......\begin{minipage}[t]{0.8\textwidth}
5\BS -4\BD 
\end{minipage}. 
 .2 3\BS......\begin{minipage}[t]{0.8\textwidth}
6\BC 
\end{minipage}. 
 .3 3NT.....\begin{minipage}[t]{0.8\textwidth}
T/P
\end{minipage}. 
 .2 4\BC......\begin{minipage}[t]{0.8\textwidth}
6\BD 
\end{minipage}. 
 .2 4\BD......\begin{minipage}[t]{0.8\textwidth}
55 MM
\end{minipage}. 
 .2 4\BH......\begin{minipage}[t]{0.8\textwidth}
54 mm, SPL \BH 
\end{minipage}. 
 .3 4NT.....\begin{minipage}[t]{0.8\textwidth}
T/P
\end{minipage}. 
 .2 4\BS......\begin{minipage}[t]{0.8\textwidth}
54 mm, SPL \BS 
\end{minipage}. 
}
\bigbreak
\subsection{Intervention}

\section{Bidding with intervention}

This chapter is about bidding with intervention in general, when we open or
they open.
\bigbreak
\subsection{Doubles}

As a rule of thumb you can say, the more your partner knows about your hand,
the more for penalties it is. Quite logical, but still.
\bigbreak
\subsubsection{Below game in competitive bidding}

Doubles are for take-out. I consider preempts also competive bidding.
\bigbreak
Examples: 1\BH -(3\BS )-D is a take-out double.
\bigbreak
\subsubsection{Game or higher}

A double is a proposition to defend.
\bigbreak
\subsubsection{Five level}

Double is for penalties and may be a Lightner double.
\bigbreak
\subsection{New suit after partners opening and an intervention}

As a general rule your new suit is forcing but not forcing for one round, so
you may pass after partners rebid.
\bigbreak
\subsection{Take care when partner is non vulnerable and preempted}

Refrain from bidding when partner may already have applied maximum
pressure non-vulnerable. You may only bid if you think you may make it.
\bigbreak
\subsection{Reverses into a suit not promised by partner are strong as usual}

For example after 1\BC -(1\BS )-D-(2\BS ) you can bid 3\BH\ with a normal opening
(although not too bad). But 3\BD\ is a reverse since partner did not promise 4
cards in diamonds.
\bigbreak
\section{They open the bidding}

This chapter is about our defensive bidding if the opponent opens something on the 1 level.
\bigbreak
\subsection{1X}

\subsubsection{Simple overcalls}

No taboos, preferably a good suit and maybe 4 cards on the 1 level.
\bigbreak
\paragraph{Fit responses}

A simple fit bid shows about 8-11 HCP, stronger than usual thus.
\bigbreak
A jump in their suit suit shows 4 cards fit and an opening.
\bigbreak
A jump fit bid in a major shows 4 trumps, a singleton and about 10-11 HCP.
\bigbreak
\paragraph{Non-fit responses}

A change of suit is non forcing at the one level. 1NT and 2NT are natural,
something like 8-11 or 12-15 HCP respectively.
\bigbreak
\subsubsection{1NT intervention}

This shows the same kind of hand as a 1NT opening albeit a little bit stronger
and usually with a stopper in their suit.
\bigbreak
The responses hereafter are just like after a 1NT opening we ignore their
bid(s).
\bigbreak
\subsection{1NT Opening}

The meaning of the bids remains unchanged when you are in second or fourth position.
\bigbreak
The meaning of our dbl depends on the strength of their 1NT opening. If the
lower limit is at most 13 HCP we consider it a weak NT. So a 14-16 NT is a
strong NT. After a strong 1NT (or if the doubler did not open) we play double
as 5+m-4M else it is for penalties.
\bigbreak
\
\dirtree{%%
 .1 (1NT). 
 .2 Dbl.....\begin{minipage}[t]{0.8\textwidth}
bicolor Mm (at least 4 cards) or strong (how many M, how many m, ?)
\end{minipage}. 
 .3 2M......\begin{minipage}[t]{0.8\textwidth}
NAT, NF
\end{minipage}. 
 .3 2\BD......\begin{minipage}[t]{0.8\textwidth}
at least 3 cards in both majors
\end{minipage}. 
 .3 2\BC......\begin{minipage}[t]{0.8\textwidth}
else
\end{minipage}. 
 .2 2\BC......\begin{minipage}[t]{0.8\textwidth}
Landy, 5(4)+4+MM
\end{minipage}. 
 .3 2\BD......\begin{minipage}[t]{0.8\textwidth}
no preference, the difference in the majors is at most 1 and may be a (light) INV
\end{minipage}. 
 .3 2M......\begin{minipage}[t]{0.8\textwidth}
Pref NF
\end{minipage}. 
 .3 2NT.....\begin{minipage}[t]{0.8\textwidth}
INV+ answers like Multi
\end{minipage}. 
 .3 3m......\begin{minipage}[t]{0.8\textwidth}
NF
\end{minipage}. 
 .3 3M......\begin{minipage}[t]{0.8\textwidth}
(light) INV
\end{minipage}. 
 .2 2\BD/2M...\begin{minipage}[t]{0.8\textwidth}
NAT
\end{minipage}. 
 .2 2NT.....\begin{minipage}[t]{0.8\textwidth}
5+5+ minors wide range
\end{minipage}. 
 .2 3m......\begin{minipage}[t]{0.8\textwidth}
wide ranged, NAT
\end{minipage}. 
 .2 3M......\begin{minipage}[t]{0.8\textwidth}
PRE
\end{minipage}. 
}
\bigbreak
\subsection{2\pdfd\ Multi-coloured}

Double is Italien (?).
\bigbreak
I personnaly prefer a Polish double (maybe the same). A double on 2\BD\ shows a
take-out on spades (short spades). A pass followed by double is a take-out on
hearts (shows shortness in hearts).
\bigbreak
\subsection{2NT Opening}

\
\dirtree{%%
 .1 (2NT). 
 .2 Dbl.....\begin{minipage}[t]{0.8\textwidth}
MM (from both hands)
\end{minipage}. 
}
\bigbreak
\subsection{Michaels}

We play two-suited overcalls after an opponents 1 level opening. The style
depends a lot on vulnerability. A two-suited overcall always shows at least 5+
in both suits.
\bigbreak
\subsubsection{Direct two-suited bids}

\
\dirtree{%%
 .1 (1\BC). 
 .2 2\BC......\begin{minipage}[t]{0.8\textwidth}
NAT
\end{minipage}. 
 .2 2\BD......\begin{minipage}[t]{0.8\textwidth}
MM
\end{minipage}. 
 .2 2NT.....\begin{minipage}[t]{0.8\textwidth}
\BD +\BH 
\end{minipage}. 
 .2 3\BC......\begin{minipage}[t]{0.8\textwidth}
Weak, 6+\BC 
\end{minipage}. 
}
\bigbreak
\
\dirtree{%%
 .1 (1\BD). 
 .2 2\BD......\begin{minipage}[t]{0.8\textwidth}
MM
\end{minipage}. 
 .2 2NT.....\begin{minipage}[t]{0.8\textwidth}
\BC +\BH 
\end{minipage}. 
 .2 3\BD......\begin{minipage}[t]{0.8\textwidth}
Asks stop for 3NT
\end{minipage}. 
}
\bigbreak
\
\dirtree{%%
 .1 (1\BH). 
 .2 2\BH......\begin{minipage}[t]{0.8\textwidth}
\BS +m
\end{minipage}. 
 .2 2NT.....\begin{minipage}[t]{0.8\textwidth}
\BC +\BD 
\end{minipage}. 
 .2 3\BH......\begin{minipage}[t]{0.8\textwidth}
Asks stop for 3NT
\end{minipage}. 
}
\bigbreak
\
\dirtree{%%
 .1 (1\BS). 
 .2 2\BS......\begin{minipage}[t]{0.8\textwidth}
\BH +m
\end{minipage}. 
 .2 2NT.....\begin{minipage}[t]{0.8\textwidth}
\BC +\BD 
\end{minipage}. 
 .2 3\BS......\begin{minipage}[t]{0.8\textwidth}
Ask stop for 3NT
\end{minipage}. 
}
\bigbreak
\paragraph{Continuations after our two-suited overcall}

\
\dirtree{%%
 .1 (1\BC)-2\BD-(P). 
 .2 Pass....\begin{minipage}[t]{0.8\textwidth}
at own risk
\end{minipage}. 
 .2 2\BH......\begin{minipage}[t]{0.8\textwidth}
NF, preference
\end{minipage}. 
 .2 2\BS......\begin{minipage}[t]{0.8\textwidth}
NF, preference
\end{minipage}. 
 .2 2NT.....\begin{minipage}[t]{0.8\textwidth}
INV+, ASK
\end{minipage}. 
 .3 3\BC......\begin{minipage}[t]{0.8\textwidth}
min/med
\end{minipage}. 
 .4 3\BD......\begin{minipage}[t]{0.8\textwidth}
asks shortness
\end{minipage}. 
 .4 3M......\begin{minipage}[t]{0.8\textwidth}
NF INV
\end{minipage}. 
 .3 3\BD......\begin{minipage}[t]{0.8\textwidth}
MAX, short \BD\ (changed 1-11-2017)
\end{minipage}. 
 .3 3\BH......\begin{minipage}[t]{0.8\textwidth}
MAX, short \BC\ (changed 1-11-2017)
\end{minipage}. 
 .3 3\BS......\begin{minipage}[t]{0.8\textwidth}
MAX, 1-1 minors
\end{minipage}. 
 .2 3\BC......\begin{minipage}[t]{0.8\textwidth}
NF, (6)7+\BC 
\end{minipage}. 
 .2 3\BD......\begin{minipage}[t]{0.8\textwidth}
NF, (6)7+\BD 
\end{minipage}. 
 .2 3\BH......\begin{minipage}[t]{0.8\textwidth}
NF, (3)4+\BH , light INV
\end{minipage}. 
 .2 3\BS......\begin{minipage}[t]{0.8\textwidth}
NF, (3)4+\BS , light INV
\end{minipage}. 
 .2 3NT.....\begin{minipage}[t]{0.8\textwidth}
T/P
\end{minipage}. 
 .2 4\BC......\begin{minipage}[t]{0.8\textwidth}
S/T \BH 
\end{minipage}. 
 .2 4\BD......\begin{minipage}[t]{0.8\textwidth}
S/T \BS 
\end{minipage}. 
 .2 4M......\begin{minipage}[t]{0.8\textwidth}
T/P
\end{minipage}. 
}
\bigbreak
\
\dirtree{%%
 .1 (1\BC)-2NT-(P). 
 .2 3\BC......\begin{minipage}[t]{0.8\textwidth}
INV+, \BH 
\end{minipage}. 
 .2 3\BD......\begin{minipage}[t]{0.8\textwidth}
NF, preference
\end{minipage}. 
 .2 3\BH......\begin{minipage}[t]{0.8\textwidth}
NF, preference
\end{minipage}. 
 .2 3\BS......\begin{minipage}[t]{0.8\textwidth}
NF, 6+\BS 
\end{minipage}. 
 .2 4\BC......\begin{minipage}[t]{0.8\textwidth}
INV, \BC 
\end{minipage}. 
 .2 4\BD......\begin{minipage}[t]{0.8\textwidth}
K/B, \BC 
\end{minipage}. 
 .2 4\BH......\begin{minipage}[t]{0.8\textwidth}
T/P
\end{minipage}. 
 .2 4\BS......\begin{minipage}[t]{0.8\textwidth}
T/P
\end{minipage}. 
}
\bigbreak
\
\dirtree{%%
 .1 (1\BD)-2\BD-(P). 
 .2 2M......\begin{minipage}[t]{0.8\textwidth}
NF preference
\end{minipage}. 
 .2 2NT.....\begin{minipage}[t]{0.8\textwidth}
INV+ ASK
\end{minipage}. 
 .3 3\BC......\begin{minipage}[t]{0.8\textwidth}
min/med
\end{minipage}. 
 .4 3\BD......\begin{minipage}[t]{0.8\textwidth}
asks shortness
\end{minipage}. 
 .4 3M......\begin{minipage}[t]{0.8\textwidth}
NF INV
\end{minipage}. 
 .3 3\BD......\begin{minipage}[t]{0.8\textwidth}
MAX, short \BC 
\end{minipage}. 
 .3 3\BH......\begin{minipage}[t]{0.8\textwidth}
MAX, short \BD 
\end{minipage}. 
 .3 3\BS......\begin{minipage}[t]{0.8\textwidth}
MAX, 1-1 minors
\end{minipage}. 
 .2 3\BC......\begin{minipage}[t]{0.8\textwidth}
NF, (6)7+\BC 
\end{minipage}. 
 .2 3\BD......\begin{minipage}[t]{0.8\textwidth}
INV, MM
\end{minipage}. 
 .2 3\BH......\begin{minipage}[t]{0.8\textwidth}
NF, (3)4+\BH , light INV
\end{minipage}. 
 .2 3\BS......\begin{minipage}[t]{0.8\textwidth}
NF, (3)4+\BS , light INV
\end{minipage}. 
 .2 3NT.....\begin{minipage}[t]{0.8\textwidth}
T/P
\end{minipage}. 
 .2 4\BC......\begin{minipage}[t]{0.8\textwidth}
S/T, \BH 
\end{minipage}. 
 .2 4\BD......\begin{minipage}[t]{0.8\textwidth}
S/T, \BS 
\end{minipage}. 
 .2 4M......\begin{minipage}[t]{0.8\textwidth}
T/P
\end{minipage}. 
}
\bigbreak
\
\dirtree{%%
 .1 (1\BD)-2NT-(P). 
 .2 3\BC......\begin{minipage}[t]{0.8\textwidth}
NF, preference
\end{minipage}. 
 .2 3\BD......\begin{minipage}[t]{0.8\textwidth}
INV(+), \BH 
\end{minipage}. 
 .2 3\BH......\begin{minipage}[t]{0.8\textwidth}
NF, preference
\end{minipage}. 
 .2 3\BS......\begin{minipage}[t]{0.8\textwidth}
NF, 6+\BS 
\end{minipage}. 
 .2 4\BC......\begin{minipage}[t]{0.8\textwidth}
INV, \BC 
\end{minipage}. 
 .2 4\BD......\begin{minipage}[t]{0.8\textwidth}
K/B, \BC 
\end{minipage}. 
 .2 4\BH......\begin{minipage}[t]{0.8\textwidth}
T/P
\end{minipage}. 
 .2 4\BS......\begin{minipage}[t]{0.8\textwidth}
T/P
\end{minipage}. 
}
\bigbreak
\
\dirtree{%%
 .1 (1\BH)-2\BH-(P). 
 .2 2\BS......\begin{minipage}[t]{0.8\textwidth}
NF, preference
\end{minipage}. 
 .2 2NT.....\begin{minipage}[t]{0.8\textwidth}
INV+, see continuation after Muiderberg
\end{minipage}. 
 .2 3\BC......\begin{minipage}[t]{0.8\textwidth}
P/C
\end{minipage}. 
 .2 3\BD......\begin{minipage}[t]{0.8\textwidth}
INV, \BS 
\end{minipage}. 
 .2 3\BH......\begin{minipage}[t]{0.8\textwidth}
S/T, \BS 
\end{minipage}. 
 .2 3\BS......\begin{minipage}[t]{0.8\textwidth}
light INV, \BS 
\end{minipage}. 
 .2 3NT.....\begin{minipage}[t]{0.8\textwidth}
T/P
\end{minipage}. 
 .2 4\BC......\begin{minipage}[t]{0.8\textwidth}
\BC +\BS 
\end{minipage}. 
 .2 4\BD......\begin{minipage}[t]{0.8\textwidth}
\BD +\BS 
\end{minipage}. 
 .2 4\BH......\begin{minipage}[t]{0.8\textwidth}
SPL for \BS 
\end{minipage}. 
 .2 4\BS......\begin{minipage}[t]{0.8\textwidth}
T/P
\end{minipage}. 
 .2 4NT.....\begin{minipage}[t]{0.8\textwidth}
bid your m
\end{minipage}. 
}
\bigbreak
\
\dirtree{%%
 .1 (1\BH)-2NT-(P). 
 .2 3m......\begin{minipage}[t]{0.8\textwidth}
NF, preference
\end{minipage}. 
 .2 3\BH......\begin{minipage}[t]{0.8\textwidth}
FG
\end{minipage}. 
 .2 3\BS......\begin{minipage}[t]{0.8\textwidth}
NF, 6+\BS 
\end{minipage}. 
 .2 3NT.....\begin{minipage}[t]{0.8\textwidth}
T/P
\end{minipage}. 
 .2 4\BC......\begin{minipage}[t]{0.8\textwidth}
INV, \BC 
\end{minipage}. 
 .2 4\BD......\begin{minipage}[t]{0.8\textwidth}
INV, \BD 
\end{minipage}. 
 .2 4\BH......\begin{minipage}[t]{0.8\textwidth}
K/B, \BD 
\end{minipage}. 
 .2 4\BS......\begin{minipage}[t]{0.8\textwidth}
T/P
\end{minipage}. 
 .2 4NT.....\begin{minipage}[t]{0.8\textwidth}
Pick best m
\end{minipage}. 
}
\bigbreak
\
\dirtree{%%
 .1 (1\BS)-2\BS-(P). 
 .2 2NT.....\begin{minipage}[t]{0.8\textwidth}
INV+, see continuation after Muiderberg
\end{minipage}. 
 .2 3\BC......\begin{minipage}[t]{0.8\textwidth}
P/C
\end{minipage}. 
 .2 3\BD......\begin{minipage}[t]{0.8\textwidth}
INV, \BH 
\end{minipage}. 
 .2 3\BH......\begin{minipage}[t]{0.8\textwidth}
NF, pref
\end{minipage}. 
 .2 3\BS......\begin{minipage}[t]{0.8\textwidth}
S/T, \BH 
\end{minipage}. 
 .2 3NT.....\begin{minipage}[t]{0.8\textwidth}
T/P
\end{minipage}. 
 .2 4\BC......\begin{minipage}[t]{0.8\textwidth}
\BC +\BH 
\end{minipage}. 
 .2 4\BD......\begin{minipage}[t]{0.8\textwidth}
\BD +\BH 
\end{minipage}. 
 .2 4\BH......\begin{minipage}[t]{0.8\textwidth}
T/P
\end{minipage}. 
 .2 4\BS......\begin{minipage}[t]{0.8\textwidth}
K/B \BH 
\end{minipage}. 
 .2 4NT.....\begin{minipage}[t]{0.8\textwidth}
bid your m
\end{minipage}. 
}
\bigbreak
\
\dirtree{%%
 .1 (1\BS)-2NT-(P). 
 .2 3m......\begin{minipage}[t]{0.8\textwidth}
NF, preference
\end{minipage}. 
 .2 3\BH......\begin{minipage}[t]{0.8\textwidth}
NF, 6+\BH 
\end{minipage}. 
 .2 3\BS......\begin{minipage}[t]{0.8\textwidth}
FG
\end{minipage}. 
 .2 3NT.....\begin{minipage}[t]{0.8\textwidth}
T/P
\end{minipage}. 
 .2 4\BC......\begin{minipage}[t]{0.8\textwidth}
INV, \BC 
\end{minipage}. 
 .2 4\BD......\begin{minipage}[t]{0.8\textwidth}
INV, \BD 
\end{minipage}. 
 .2 4\BH......\begin{minipage}[t]{0.8\textwidth}
T/P
\end{minipage}. 
 .2 4\BS......\begin{minipage}[t]{0.8\textwidth}
?
\end{minipage}. 
 .2 4NT.....\begin{minipage}[t]{0.8\textwidth}
Pick best m
\end{minipage}. 
}
\bigbreak
\subsubsection{They bid after our two-suited overcall}

If they bid a new suit intended as natural, dbl is penalty.  If they support
each other, dbl is for take-out and may be INV for the major if you don't have
another invitational bid.  If they bid one of our suits, dbl means that you would have
liked to bid that.
\bigbreak
\section{We open the bidding}

\subsection{They intervene with a natural 1NT}

\
\dirtree{%%
 .1 1m-(1NT). 
 .2 Dbl.....\begin{minipage}[t]{0.8\textwidth}
bicolor Mm (see our intervention after 1NT)
\end{minipage}. 
 .2 2m......\begin{minipage}[t]{0.8\textwidth}
both MM
\end{minipage}. 
 .2 2om.....\begin{minipage}[t]{0.8\textwidth}
NAT
\end{minipage}. 
 .2 2M......\begin{minipage}[t]{0.8\textwidth}
NAT
\end{minipage}. 
}
\bigbreak
\subsection{1M-(D)}

We will play transfers starting from 1NT till the bid below two of our
suit. All those transfers show the next suit and they do not promise a
rebid. A direct raise is weaker than the transfer to 2M. A jump is that suit
plus support.
\bigbreak
\
\dirtree{%%
 .1 1M-(D). 
 .2 1NT.....\begin{minipage}[t]{0.8\textwidth}
TRF \BC 
\end{minipage}. 
 .2 2\BC......\begin{minipage}[t]{0.8\textwidth}
TRF \BD 
\end{minipage}. 
 .2 3m......\begin{minipage}[t]{0.8\textwidth}
support plus that suit
\end{minipage}. 
}
\bigbreak
\
\dirtree{%%
 .1 1\BH-(D). 
 .2 2\BD......\begin{minipage}[t]{0.8\textwidth}
TRF \BH\ (fit), stronger than 2\BH\ immediately
\end{minipage}. 
 .2 2\BH......\begin{minipage}[t]{0.8\textwidth}
3\BH , weaker than 2\BD\ immediately
\end{minipage}. 
}
\bigbreak
\
\dirtree{%%
 .1 1\BS-(D). 
 .2 2\BH......\begin{minipage}[t]{0.8\textwidth}
TRF \BS\ (fit), stronger than 2\BS\ immediately
\end{minipage}. 
 .2 2\BS......\begin{minipage}[t]{0.8\textwidth}
3\BS , weaker than 2\BH\ immediately
\end{minipage}. 
}
\bigbreak
\subsection{fit after 1M and intervention below 2NT}

The bid of 2NT shows at least four cards fit and at least an invitational hand, for
example 1\BH -(2\BH )-2NT. There is one exception: when the cue bid is above 3M,
2NT just shows at least three cards fit like 1\BH -(2\BS )-2NT.
\bigbreak
\section{Transfer Lebensohl}

We play this after our 1 NT opening and their intervention of 2\BC\ till
2\BS . And also after their weak two level opening bids, whether it be some kind
or multi or not.
\bigbreak
\subsection{One suit known}

Transfer Lebensohl starts at 2NT. There are 4 types of special bids:
\bigbreak
\begin{enumerate}
\item A transfer to their suit which is FG and shows at 4 cards in at least one
   of the unbid majors.

\item 2NT, a Puppet to 3\BC . It can be either a sign-off below their suit or FG
   with at least 5 clubs.

\item A transfer to a suit. This is at least invitational and
   shows 5 cards in the transfer suit unless it is a transfer in a major which
   has been implied by partner's take-out double. It does not matter whether the transfer
   is above or below their suit.

\item 3\BS . This is FG and denies a stop and it denies a 4 card major in an unbid
   suit.

\end{enumerate}
\bigbreak
\
\dirtree{%%
 .1 1NT. 
 .2 (2X)....\begin{minipage}[t]{0.8\textwidth}
5+X
\end{minipage}. 
}
\bigbreak
\
\dirtree{%%
 .1 1NT-(2\BC). 
 .2 2X......\begin{minipage}[t]{0.8\textwidth}
NAT, S/O
\end{minipage}. 
 .2 2NT.....\begin{minipage}[t]{0.8\textwidth}
A transfer to their suit, hence FG and at least one 4 card major
\end{minipage}. 
 .3 3\BC......\begin{minipage}[t]{0.8\textwidth}
I do \textbf{not} have a stopper
\end{minipage}. 
 .3 3M......\begin{minipage}[t]{0.8\textwidth}
I do have a stopper as well as 4 cards in this major
\end{minipage}. 
 .2 3\BC......\begin{minipage}[t]{0.8\textwidth}
A transfer to \BD , INV+, 5+\BD 
\end{minipage}. 
 .2 3\BD......\begin{minipage}[t]{0.8\textwidth}
A transfer to \BH , INV+, 5+\BH 
\end{minipage}. 
 .2 3\BH......\begin{minipage}[t]{0.8\textwidth}
A transfer to \BS , INV+, 5+\BS 
\end{minipage}. 
 .2 3\BS......\begin{minipage}[t]{0.8\textwidth}
FG, no stopper, no 4 card major
\end{minipage}. 
}
\bigbreak
\
\dirtree{%%
 .1 1NT-(2\BD). 
 .2 2X......\begin{minipage}[t]{0.8\textwidth}
NAT, S/O
\end{minipage}. 
 .2 2NT.....\begin{minipage}[t]{0.8\textwidth}
PUP
\end{minipage}. 
 .3 3\BC......\begin{minipage}[t]{0.8\textwidth}
Forced
\end{minipage}. 
 .4 Pass....\begin{minipage}[t]{0.8\textwidth}
5+\BC , S/O
\end{minipage}. 
 .4 3\BD......\begin{minipage}[t]{0.8\textwidth}
FG, 5+\BC , no 4M, asks primarily for stopper but different from 3\BS\ immediately
\end{minipage}. 
 .4 3M......\begin{minipage}[t]{0.8\textwidth}
FG, 5+\BC , 4M
\end{minipage}. 
 .2 3\BC......\begin{minipage}[t]{0.8\textwidth}
A transfer to their suit, hence FG and at least one 4 card major
\end{minipage}. 
 .3 3\BD......\begin{minipage}[t]{0.8\textwidth}
I do \textbf{not} have a stopper
\end{minipage}. 
 .3 3M......\begin{minipage}[t]{0.8\textwidth}
I do have a stopper as well as 4 cards in this major
\end{minipage}. 
 .2 3\BD......\begin{minipage}[t]{0.8\textwidth}
A transfer to \BH , INV+, 5+\BH 
\end{minipage}. 
 .2 3\BH......\begin{minipage}[t]{0.8\textwidth}
A transfer to \BS , INV+, 5+\BS 
\end{minipage}. 
 .2 3\BS......\begin{minipage}[t]{0.8\textwidth}
FG, no stopper, no 4 card major
\end{minipage}. 
}
\bigbreak
\
\dirtree{%%
 .1 1NT-(2\BH). 
 .2 2X......\begin{minipage}[t]{0.8\textwidth}
NAT, S/O
\end{minipage}. 
 .2 2NT.....\begin{minipage}[t]{0.8\textwidth}
PUP
\end{minipage}. 
 .3 3\BC......\begin{minipage}[t]{0.8\textwidth}
Forced
\end{minipage}. 
 .4 Pass....\begin{minipage}[t]{0.8\textwidth}
5+\BC , S/O
\end{minipage}. 
 .4 3\BD......\begin{minipage}[t]{0.8\textwidth}
5+\BD , S/O
\end{minipage}. 
 .4 3\BH......\begin{minipage}[t]{0.8\textwidth}
FG, 5+\BC , no 4\BS , asks primarily for stopper but different from 3\BS\ immediately
\end{minipage}. 
 .4 3\BS......\begin{minipage}[t]{0.8\textwidth}
FG, 5+\BC , 4\BS 
\end{minipage}. 
 .2 3\BC......\begin{minipage}[t]{0.8\textwidth}
A transfer to \BD , INV+, 5+\BD 
\end{minipage}. 
 .2 3\BD......\begin{minipage}[t]{0.8\textwidth}
A transfer to their suit, hence FG and 4\BS 
\end{minipage}. 
 .3 3\BH......\begin{minipage}[t]{0.8\textwidth}
I do \textbf{not} have a stopper and probably not 4\BS\ as well
\end{minipage}. 
 .3 3\BS......\begin{minipage}[t]{0.8\textwidth}
4\BS , with or without stopper
\end{minipage}. 
 .2 3\BH......\begin{minipage}[t]{0.8\textwidth}
A transfer to \BS , INV+, 5+\BS 
\end{minipage}. 
 .2 3\BS......\begin{minipage}[t]{0.8\textwidth}
FG, no stopper, no 4\BS 
\end{minipage}. 
}
\bigbreak
\
\dirtree{%%
 .1 1NT-(2\BS). 
 .2 2NT.....\begin{minipage}[t]{0.8\textwidth}
PUP
\end{minipage}. 
 .3 3\BC......\begin{minipage}[t]{0.8\textwidth}
Forced
\end{minipage}. 
 .4 Pass....\begin{minipage}[t]{0.8\textwidth}
5+\BC , S/O
\end{minipage}. 
 .4 3\BD......\begin{minipage}[t]{0.8\textwidth}
5+\BD , S/O
\end{minipage}. 
 .4 3\BH......\begin{minipage}[t]{0.8\textwidth}
5+\BH , S/O
\end{minipage}. 
 .4 3\BS......\begin{minipage}[t]{0.8\textwidth}
FG, 5+\BC , no 4\BH , asks primarily for stopper but different from 3\BS\ immediately
\end{minipage}. 
 .2 3\BC......\begin{minipage}[t]{0.8\textwidth}
A transfer to \BD , INV+, 5+\BD 
\end{minipage}. 
 .2 3\BD......\begin{minipage}[t]{0.8\textwidth}
A transfer to \BS , INV+, 5+\BH 
\end{minipage}. 
 .2 3\BH......\begin{minipage}[t]{0.8\textwidth}
A transfer to their suit, hence FG and 4\BH 
\end{minipage}. 
 .3 3\BS......\begin{minipage}[t]{0.8\textwidth}
I do \textbf{not} have a stopper and not 4\BH\ as well
\end{minipage}. 
 .2 3\BS......\begin{minipage}[t]{0.8\textwidth}
FG, no stopper, no 4\BH 
\end{minipage}. 
}
\bigbreak
\subsection{Two suits known}

\
\dirtree{%%
 .1 1NT-(2\BC)\begin{minipage}[t]{0.8\textwidth}
both MM
\end{minipage}. 
 .2 Dbl.....\begin{minipage}[t]{0.8\textwidth}
take-out (with balanced hands)
\end{minipage}. 
 .2 Pass....\begin{minipage}[t]{0.8\textwidth}
maybe a trap pass (a later double is for penalties, the usual method for dealing with twosuiters)
\end{minipage}. 
 .2 2X......\begin{minipage}[t]{0.8\textwidth}
NAT, S/O (also 2\BH\ and 2\BS\ unless this is a known 5 card)
\end{minipage}. 
 .2 2NT.....\begin{minipage}[t]{0.8\textwidth}
A transfer to \BC , either S/O or FG
\end{minipage}. 
 .2 3\BC......\begin{minipage}[t]{0.8\textwidth}
A transfer to \BD , INV+ (with a weak hand you bid 2\BD )
\end{minipage}. 
 .2 3\BD......\begin{minipage}[t]{0.8\textwidth}
A transfer to \BH , hence SPL and FG
\end{minipage}. 
 .2 3\BH......\begin{minipage}[t]{0.8\textwidth}
A transfer to \BS , hence SPL and FG
\end{minipage}. 
 .2 3\BS......\begin{minipage}[t]{0.8\textwidth}
FG, no stopper in \BH\ nor \BS 
\end{minipage}. 
}
\bigbreak
\subsection{Multi-coloured}

After Multi we play that double shows short spades and at least 3 cards in
hearts. This is a Polish convention and it allows us to know better what is
going on. We will assume that after a double their suit is spades.
\bigbreak
\
\dirtree{%%
 .1 (2\BD)-D..\begin{minipage}[t]{0.8\textwidth}
take-out on \BS , hence short \BS\ (pass first with short \BH )
\end{minipage}. 
 .2 (P). 
 .3 Pass....\begin{minipage}[t]{0.8\textwidth}
long diamonds if pass promises \BD\ (always ask explanation)
\end{minipage}. 
 .3 2\BH......\begin{minipage}[t]{0.8\textwidth}
NAT, NF
\end{minipage}. 
 .3 2\BS......\begin{minipage}[t]{0.8\textwidth}
6\BS , NAT, NF (try to pass though)
\end{minipage}. 
 .3 2NT.....\begin{minipage}[t]{0.8\textwidth}
PUP, either S/O in \BC\ (you can pass with long \BD )
\end{minipage}. 
 .3 3\BC......\begin{minipage}[t]{0.8\textwidth}
A transfer to \BD , INV+{.} If their pass shows long diamonds this is a cuebid, see 1NT-2\BD\ natural
\end{minipage}. 
 .3 3\BD......\begin{minipage}[t]{0.8\textwidth}
A transfer to \BH {.} If FG it shows 5+\BH\ else 4+\BH\ (partner promises 3+\BH )
\end{minipage}. 
 .3 3\BH......\begin{minipage}[t]{0.8\textwidth}
A transfer to \BS , a kind of cue bid hence FG and it shows 4\BH 
\end{minipage}. 
 .2 (2M). 
 .3 2NT.....\begin{minipage}[t]{0.8\textwidth}
PUP, either S/O in a minor else FG with 5+\BC 
\end{minipage}. 
 .3 3\BC......\begin{minipage}[t]{0.8\textwidth}
A transfer to \BD , INV+
\end{minipage}. 
 .3 3\BD\BH\BS....\begin{minipage}[t]{0.8\textwidth}
see (2\BD )-D-(P)
\end{minipage}. 
}
\bigbreak
\section{Defense against two-suiters}

The cheapest bid in of their suits shows the fourth suit and is forcing to
game. Thus not the lowest in the sense of clubs, diamonds but the most
economical bid. The most expensive bid in one of their suits shows fit and is
at least invitational. The reason for this scheme is that you need more space
if you do not have a fit. A 2NT bid shows fit with at least 4 cards and is at
least invitational.
\bigbreak
\section{Slem bidding}

\subsection{Game forcing fit situations}

\subsubsection{Major at the three level}

For example 1\BS -2\BH -3\BH .
\bigbreak
When a fit has been agreed upon, one may sign-off in four of the major as a
sign-off. 3NT is not a minimum but not very strong neither ("la première
zone") and a control bid is stronger ("la seconde zone").
\bigbreak
\subsection{Blackwood}

Actually Roman Key Card Blackwood with old fashioned responses (30-41).
\bigbreak
\
\dirtree{%%
 .1 4NT. 
 .2 5\BC......\begin{minipage}[t]{0.8\textwidth}
0-3 key cards
\end{minipage}. 
 .2 5\BD......\begin{minipage}[t]{0.8\textwidth}
1-4 key cards
\end{minipage}. 
 .2 5\BH......\begin{minipage}[t]{0.8\textwidth}
2-5 key cards, no trump Queen
\end{minipage}. 
 .2 5\BS......\begin{minipage}[t]{0.8\textwidth}
2-5 key cards, trump Queen
\end{minipage}. 
}
\bigbreak
\subsubsection{Asking for the trump Queen}

The trump Queen can be demanded after 5\BC /5\BD\ with the first free bid. 
The lowest of 5NT and the trump suit without jump denies the Queen. Any other
bid in a suit promises the Queen plus only the King in that suit OR the two
other Kings (the King of trumps is ignored of course).
\bigbreak
\subsubsection{Intervention after 4NT}

Double (or redouble) is for penalties. Pass shows an even number : 0/2/4 key
cards. The first free bid shows 1/3/5 key cards.
\bigbreak
\section{Abbreviations}

An excerpt from http://www.worldbridge.org/wp-content/uploads/2017/04/Guidetocompletion.pdf.
\bigbreak
The following abbreviations or terms may be used (Note the use of BLOCK
CAPITALS and SLASHES (/)):
\bigbreak
\begin{itemize}
\item (5431)       = Any hand with that distribution (suits unknown)

\item 5431         = Five spades, four hearts, three diamonds, one club

\item 54(31)       = A hand with five spades, four hearts, and 3\BD 1\BC\ or 3\BC 1\BD\ 

\item 54(xx)       = A hand with five spades and four hearts

\item AGG          = Aggressor. The first player to double or overcall for the defending side

\item ADV          = Advancer, aggressors partner

\item ASK          = Asking bid

\item ART          = Artificial

\item ATT          = Attitude

\item B            = Black suit(s)

\item BAL          = Balanced

\item BW           = Blackwood

\item CB           = Checkback

\item COMP         = Competitive

\item CONC         = Concentrated (e.g. all values in the bid suits)

\item CONST        = Constructive

\item CTRL         = Control

\item CUE          = Cue-bid

\item DISC         = Discourage (ing)

\item E            = Even

\item ENC          = Encourage (ing)

\item FRAG         = Fragment

\item F            = Forcing

\item F1           = Forcing 1 round

\item F2NT         = Forcing to 2NT

\item FG           = Forcing to game

\item 4SF          = 4th suit forcing (4SFG, 4SF1)

\item FREQ         = Frequent

\item G/T          = Game try

\item H            = Honour (Ace, King, or Queen)

\item HCP          = High Card Points

\item INV          = Invitational

\item INQ          = Inquiry

\item KCB          = Keycard Blackwood

\item L/D          = Lead-directing

\item LEB          = lebensohl

\item LHO          = The opponent on your left

\item LIM          = Limit raise

\item L/S          = Long suit

\item L/T          = Less than (length or strength)

\item M            = Major

\item MM           = majors

\item m            = Minor

\item mm           = minors

\item MAX          = Maximum, Maximal, Maximal Overcall Double

\item MIN          = Minimum

\item NAT          = Natural

\item NEG          = Negative

\item NEU          = Neutral

\item NF           = Nonforcing

\item NT           = No Trump

\item NV           = Nonvulnerable

\item oM           = The other major

\item om           = The other minor

\item OPPT         = Opponent(s)

\item OPT          = Optional

\item O/S          = Outside

\item O/C          = Overcall

\item P/C          = Pass or correct

\item PEN          = Penalty

\item PH           = Passed hand

\item PRE          = Pre-emptive

\item PUP          = Puppet to (e.g. 2\BC\ demands 2\BD )

\item QUANT        = Quantitative

\item (R)          = Relay (e.g. 2\BC\ asks for shape description etc)

\item RESP         = Responder; Response; Responsive

\item REV          = Reverse

\item RHO          = The opponent on your right

\item RKCB         = Roman Keycard Blackwood

\item R/O          = Reopening

\item S/P          = Suit preference

\item S/A          = Suit agreement

\item S/O          = Signoff, shutout

\item SOL          = Solid (suit)

\item S-SOL        = Semi-solid (suit)

\item SPL          = Splinter, or short suit

\item S/S          = Short suit

\item S/T          = Slam try

\item STAY         = Stayman

\item STR          = Strong

\item SUPP         = Support

\item T/O          = Takeout

\item TRF          = Transfer

\item UNT          = Unusual No Trump

\item VUL or V     = Vulnerable

\item w/           = With

\item w/o          = Without

\item WJO          = Weak jump overcall

\item WJS          = Weak jump shift

\item WK           = Weak

\end{itemize}
\bigbreak
Besides these abbreviations from the WBF, the following abbreviations have been added:
\bigbreak
\begin{itemize}
\item K/B          = (Optional) KickBack

\item MED          = medium

\item T/P          = To Play

\item UNB          = Unbalanced

\item CoG          = Choice of Games

\end{itemize}
\bigbreak
The following abbreviations are not conform the WBF, but conform the BML definitions:
\bigbreak
\begin{itemize}
\item D            = Double

\item R            = Redouble (hence not the reds)

\end{itemize}
\bigbreak
The following abbreviations are \textbf{not} used:
\bigbreak
\begin{itemize}
\item DBL or X     = Double

\item RDBL, RD, XX = Redouble

\end{itemize}
\bigbreak
\end{document}
